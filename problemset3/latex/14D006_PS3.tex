\documentclass[11pt, oneside]{article}   	% use "amsart" instead of "article" for AMSLaTeX format
\usepackage{geometry}                		% See geometry.pdf to learn the layout options. There are lots.
\geometry{letterpaper}                   		% ... or a4paper or a5paper or ...
%\geometry{landscape}                		% Activate for rotated page geometry
%\usepackage[parfill]{parskip}    		% Activate to begin paragraphs with an empty line rather than an indent
\usepackage{graphicx}				% Use pdf, png, jpg, or eps§ with pdflatex; use eps in DVI mode
								% TeX will automatically convert eps --> pdf in pdflatex
								\usepackage[bb=boondox]{mathalfa}
\usepackage{mathtools}
\usepackage{amssymb}

\usepackage{amsmath,amsfonts,amsthm} % Math packages
\usepackage{bm}
\usepackage{graphicx}
\usepackage{dsfont}
\graphicspath{ {images/} }

%SetFonts

%SetFonts

\DeclareMathOperator*{\argmin}{arg\,min}
\DeclareMathOperator*{\argmax}{arg\,max}
\newcommand{\horrule}[1]{\rule{\linewidth}{#1}} % Create horizontal rule command with 1 argument of height

\title{
\normalfont \normalsize
\textsc{14D006 Stochastic Models and Optimization} \\ [25pt] % Your university, school and/or department name(s)
\horrule{0.5pt} \\[0.4cm] % Thin top horizontal rule
\huge Problemset 3\\ % The assignment title
\horrule{2pt} \\[0.5cm] % Thick bottom horizontal rule
}

\author{Daniel Bestard, Michael Cameron, Hans-Peter H{\"o}llwirth, Akhil Lohia} % Your name

\date{\normalsize\today} % Today's date or a custom date

\begin{document}
\maketitle


%%%%%%%%
% Problem 1 %
%%%%%%%%
\section{Inventory Control with Forecasts}


%%%%%%%%
% Problem 2 %
%%%%%%%%
\section{Inventory Pooling}
We want to compare the optimal expected cost between a decentralized and a pooled inventory system with $n$ locations. Let the demand at each location $i$ be independently, normally distributed with mean $\mu$ and variance $\sigma^2$, i.e. $D_1,...D_n \stackrel{iid}{\sim} N(\mu, \sigma^2)$. Then $\sum_{i=1}^n D_i \stackrel{d}{=} \sqrt{n}D_1 + \mu (n-\sqrt{n})$. \\


Let $h$ and $b$ denote the inventory holding cost and backorder cost, respectively. We already know that if $G(Q^*) = \mathbb{E}_D[h(Q^*-D)_+ + b(D-Q^*)_+]$ denotes the optimal expected cost for each location (with optimal order quantity $Q^* = \inf \left\{ Q \geq 0: \mathbb{P}\left(D_1 \leq Q \right) \geq \frac{b}{b+h} \right\}$), then the optimal expected cost in the decentralized inventory system is
$$G_D(Q^*) = \sum_{i=1}^n G(Q^*) = n G(Q^*)$$

To find the optimal expected cost in the pooled inventory system $G_P(Q_P^*)$, we first need to determine the optimal order quantity $Q_P^*$ under this system. 

\begin{align*}
Q_P^* &=  \inf \left\{ Q \geq 0: \mathbb{P}\left( \sum_{i=1}^n D_i \leq Q \right) \geq \frac{b}{b+h} \right\}\\
&=  \inf \left\{ Q \geq 0: \mathbb{P}\left( \sqrt{n}D_1 + \mu (n-\sqrt{n}) \leq Q \right) \geq \frac{b}{b+h} \right\}\\
&=  \inf \left\{ \sqrt{n}X + \mu (n-\sqrt{n}) \geq 0: \mathbb{P}\left(D_1 \leq X \right) \geq \frac{b}{b+h} \right\}\\
&=  \sqrt{n} \inf \left\{ X \geq 0: \mathbb{P}\left(D_1 \leq X \right) \geq \frac{b}{b+h} \right\} + \mu (n-\sqrt{n})\\
&=  \sqrt{n} Q^* + \mu (n-\sqrt{n})\\
\end{align*}

With this we can now determine optimal expected cost in the pooled inventory system $G_P(Q_P^*)$:

\begin{align*}
G_P(Q_P^*) &=  \mathbb{E}_{\{D\}}\left[h(Q_P^* - \sum_{i=1}^n D_i)_+ + b(\sum_{i=1}^n D_i - Q_P^*)_+\right]\\
&=  \mathbb{E}_{\{D\}}\big[h((\sqrt{n} Q^* + \mu (n-\sqrt{n})) - (\sqrt{n}D_1 + \mu (n-\sqrt{n})))_+ \\
&+ b((\sqrt{n}D_1 + \mu (n-\sqrt{n})) - (\sqrt{n} Q^* + \mu (n-\sqrt{n})))_+\big]\\
&=  \mathbb{E}_{\{D\}}\big[h(\sqrt{n} Q^* - \sqrt{n}D_1)_+ + b(\sqrt{n}D_1 - \sqrt{n} Q^* )_+\big]\\
&=  \sqrt{n} \cdot \mathbb{E}_{\{D\}}\big[h( Q^* - D_1)_+ + b(D_1 -  Q^* )_+\big]\\
&=  \sqrt{n} G(Q^*)\\
\end{align*}

It follows that
$$
\frac{G_D(Q^*)}{G_P(Q_P^*)} = \frac{n G(Q^*)}{\sqrt{n} G(Q^*)} = \frac{n}{\sqrt{n}} = \sqrt{n} 
$$
This result means that the expected optimal cost of the decentralized system is $\sqrt{n}$-times as large as the expected optimal cost of the pooled inventory system under the given assumptions.


%%%%%%%%
% Problem 3 %
%%%%%%%%
\section{An Investment Problem}
We want to find the optimal investment strategy in a situation where the investor can make $N$ sequential investment decisions, each resulting in one of two possible outcomes: Either (1) the invested money doubles (with probability $1/2 < p < 1$, or (2) the invested money is lost (with probability $1-p$). \\

Let $x_k$ denote the wealth at time $k$ (initial wealth $x_0$), let $u_k \in U_k(x_k) = [0, x_k]$ be the investment (decision) at time $k$ and let 
$$
w_k(u_k) = 
\begin{cases} 
2 u_k & \text{ with probability } p\\
0 & \text{ with probability } (1-p)\\
\end{cases}
$$
be the outcome of the investment at time $k$. 

First, note that the state transition has the following form:
$$
x_{k+1} = f(x_k,u_k,w_k) = x_k - u_k + w_k = 
\begin{cases} 
x_k + u_k & \text{ with probability } p\\
x_k - u_k & \text{ with probability } (1-p)\\
\end{cases}
$$ 

The DP algorithm has the form (note that we can consider the logarithm of the investor's wealth after the $N^{th}$ investment as the objective function without changing the optimal strategy):

\begin{align*}
J_N(x_N) &= \ln(x_N)\\
\\
J_k(x_k) &=  \max_{u_k \in U_k(x_k)} \mathbb{E}_{w_k} \left[ J_{k+1} (x_{k+1}) \right]\\
&=  \max_{u_k \in U_k(x_k)} \left[ p J_{k+1} (x_k + u_k) + (1-p) J_{k+1} (x_k - u_k)\right]\\
\end{align*}

We want to find the strategy that yields the maximum expected wealth at time $N$, given initial wealth $x_0$: $J_0(x_0)$. \\ First, consider the optimal strategy for a one-period investment horizon, i.e. $J_0(x_0) = J_{N-1}(x_{N-1}) = \max_{u_k \in U_k(x_k)} \left[ p \ln (x_k + u_k) + (1-p) \ln (x_k - u_k)\right]$. We can find the optimal investment strategy $u_{N-1}^*$ using first-order conditions:

\begin{align*}
\frac{\partial}{\partial u_{N-1}} J_{N-1}(x_{N-1}) &= 0\\
\frac{\partial}{\partial u_{N-1}} \left[ p \ln (x_{N-1} + u_{N-1}) + (1-p) \ln (x_{N-1} - u_{N-1})\right] &= 0\\
\frac{p}{x_{N-1} + u_{N-1}^*} - \frac{1-p}{x_{N-1} - u_{N-1}^*} &= 0\\
p(x_{N-1} - u_{N-1}^*) &= (1-p)(x_{N-1} + u_{N-1}^*)\\
u_{N-1}^* &= (2p-1)x_{N-1}\\
\end{align*}

It follows that $J_{N-1}(x_{N-1})$ has the form:
\begin{align*}
J_{N-1}(x_{N-1}) &=  \max_{u_{N-1}} \mathbb{E}_{w_{N-1}} \left[\ln J_{N} (x_{N}) \right]\\
&=  \max_{u_{N-1}} \left[ p \ln (x_{N-1} + u_{N-1}) + (1-p) \ln (x_{N-1} - u_{N-1})\right]\\
&=  p \ln (x_{N-1} + (2p-1)x_{N-1}) + (1-p) \ln (x_{N-1} - (2p-1)x_{N-1})\\
&=  p \ln (2p \cdot x_{N-1}) + (1-p) \ln ((2-2p) \cdot x_{N-1})\\
&=  p \ln (2p) + \ln(x_{N-1}) + (1-p) \ln (2-2p) + \ln(x_{N-1})\\
&=  p \ln (2p) - p\ln (2-2p) + \ln (2-2p) + \ln(x_{N-1})\\
&=  A_{N-1} + \ln(x_{N-1})\\
\end{align*}
where $A_{N-1} = p \ln (2p) - p\ln (2-2p) + \ln (2-2p)$ is independent of $x_{N-1}$. We see that $J_{N-1}(x_{N-1})$ contains the term $\ln(x_{N-1})$ and so the optimal strategy (known as "Kelly strategy") generalizes to all $k=0,1,..,(N-1)$ in the multi-period problem:
$$u_{k}^* = (2p-1)x_{k}$$
and therefore $J_k(x_k) = A_{k} + \ln(x_{k})$ for some $A_K$ independent of $x_k$.





%%%%%%%%
% Problem 4 %
%%%%%%%%
\section{Asset Selling with Maintenance Cost}
Let's start by setting some notation. Let $w_{0},w_{1},...,w_{N_1}$ be the offers the household gets. Given that the problem does not specify anything about investing the money from selling the house we will assume that such money does not get any interest rate from any possible investment. Let $T$ denote the terminate state. Let $x_{k} = T$ at some time $k \leq N-1$ denote that the house has already been sold. Likewise, let $x_{k} \neq T$ at some time $k \leq N-1$ denote that the house has not been sold yet.\\

Let's now identify the variables that define the DP algorithm that we will used in this exercise:
\begin{itemize}
	\item $x_{k}$: state of the system at period $k$, which is the amount of money that the household gets at such period.
	\item $u_{k}$: control variable at period $k$. In this case the household can take two action: sell or not sell. That is $u_{k} \in \{u_{k}^{1}\text{(sell)}, u_{k}^{2}\text{(not sell)}\}$
	\item $w_{k}$: uncertainty at time $k$, which corresponds to the offers.
\end{itemize}
Taking into account the fact that the rejects offers can be accepted in the future, the dynamics of the DP algorithm are:
$$x_{k+1} = \begin{cases}
T & if \, \, x_{k}=T, \, \, or \, \, if \, \, x_{k} \neq T \, \, and \, \, u_{k}=u^{1}\\
max(x_{k},w_{k}) & otherwise
\end{cases}$$
and the reward function is
$$\underset{w_{k}}{\mathbb{E}} \bigg\{g_{N}(x_{N}) + \sum_{k=0}^{N-1} g_{k}(x_{k},u_{k},w_{k})\bigg\}$$
where
$$g_{N}(x_{N}) = \begin{cases}
				x_{N} \, \, & if \, \, x_{N} \neq T\\
				0 \, \,  & otherwise
			   \end{cases}
$$
$$g_{k}(x_{k},u_{k},w_{k}) = \begin{cases}
						x_{k} \, \, & if \, \, x_{N} \neq T \, \, and \, \, u_{k} = u^{1}\\
						-c \, \, & if \, \, x_{N} \neq T \, \, and \, \, u_{k} = u^{2}\\
				0 \, \,  &  if \, \, x_{N} = T
			   \end{cases}
$$
Therefore, the DP algorithm is,
$$J_{N}(x_{N}) = \begin{cases}
				x_{N} \, \, & if \, \, x_{N} \neq T\\
				0 \, \,  & otherwise
			   \end{cases}
$$
\begin{align*}
J_{k}(x_{k}) &= 
\begin{cases}
				\underset{u_{k} \in U_{k}(x_{k})}{max} \underset{w_{k}}{\mathbb{E}} \big\{g_{k}(x_{k},u_{k},w_{k}) + J_{k+1}(x_{k+1})\big\} \, \, & if \, \, x_{N} \neq T\\
				0 \, \,  & if \, \, x_{N} = T
			   \end{cases}\\
			   &= 
\begin{cases}
				max \underset{w_{k}}{\mathbb{E}} \big\{g_{k}(x_{k},u_{k}^{1},w_{k}) + J_{k+1}(x_{k+1}), \, g_{k}(x_{k},u_{k}^{2},w_{k}) + J_{k+1}(x_{k+1})\big\} \, \, & if \, \, x_{N} \neq T\\
				0 \, \,  & if \, \, x_{N} = T
			   \end{cases}\\
			   &=
\begin{cases}
				max\big\{\text{Revenue from selling, Expected revenue from not selling}\}\big\} \, \, & if \, \, x_{N} \neq T\\
				0 \, \,  & if \, \, x_{N} = T
			   \end{cases}\\
			   &=
			   \begin{cases}
				max\big\{x_{k} + 0,\underset{w_{k}}{\mathbb{E}}\{J_{k+1}(w_{k})-c\}\big\} \, \, & if \, \, x_{N} \neq T\\
				0 \, \,  & if \, \, x_{N} = T
			   \end{cases}
\end{align*}
This leads to the one-stopping set of the following form:
\begin{align*}
T_{N-1} &= \big\{x | x > \underset{w_{k}}{\mathbb{E}} (J_{N}(x_{N}) - c)\big\}\\
&= \big\{x | x > \underset{w_{k}}{\mathbb{E}} (max(x_{N-1},w_{N-1}) - c)\big\}\\
&= \big\{x | x >  \sum_{j=1}^{n} p_{j} max(x_{N-1},w_{j,N-1}) - c\big\}\\
&= \big\{x | x >  \sum_{j=1}^{n} p_{j} max(x,w_{j}) - c\big\}\\
\end{align*}
Note that the last line of the previous mathematical development comes from the fact that the function $\sum_{j=1}^{n} p_{j} max(x_{N-1},w_{j,N-1}) - c$ is non-increasing which implies that $T_{N-1}$ is absorbing in the sense that if a state belongs to $T_{N-1}$ and termination is not selected, then the next state will also be in $T_{N-1}$. If we define $\bar{\alpha}$ to be
$$\bar{\alpha} = \sum_{j=1}^{n} p_{j} max(x,w_{j}) - c$$
then the optimal policy is defined by,
$$\mu_{k}^{*}(x_{k}) = \begin{cases}
				u^{1} \, \, & if \, \, x_{k} > \bar{\alpha}\\
				u^{2} \, \, & if \, \, x_{k} < \bar{\alpha}\\
			   \end{cases}
$$

%%%%%%%%
% Problem 5 %
%%%%%%%%
\section{Scheduling Problem}
(a) We solve this problem using list indexing. Let $L$ be the optimal order of answering the questions. Let $i$ and $j$ be the $k^{th}$ and $(k + 1)^{st}$ questions in the optimally ordered list L.
$$
L = (i_{0},...,i_{k-1},i,j,i_{k+2},...,i_{N-1})
$$
We can then calculate the expected return for answering the questions in this order.
\begin{align*}
\mathbb{E}[{reward~of~L}] = & \mathbb{E} [reward~of~\{i_{0},...,i_{k-1}\}] \\
+ & p_{i_{0}}···p_{i_{k-1}} p_{i}(R_{i}+p_{j}R_{j} -(1-p_{j})F_{j})-(1-p_{i})F_{i}  \\
+ & p_{i_{0}} ···p_{i_{k}} p_{i}p_{j} \mathbb{E}[reward~of~\{i_{k+2},...,i_{N-1}\}] \\
\end{align*}
Now we consider the is were quiestions $i$ and $j$ are interchanged.
$$
L^{\prime} = (i_{0},...,i_{k-1},j,i,i_{k+2},...,i_{N-1})
$$
Since $L$ is optimal, $\mathbb{E}[{reward~of~L}] \geq \mathbb{E}[{reward~of~L^{\prime}}]$
This is equivalent to 
\begin{align*}
p_{i}(R_{i}+p_{j}R_{j} -(1-p_{j})F_{j})-(1-p_{i})F_{i} &\geq p_{j}(R_{j}+p_{i}R_{i} -(1-p_{i})F_{i})-(1-p_{j})F_{j} \\
\implies \frac{p_{i}R_{i}-(1-p_{i})F_{i}}{1-p_{i}} &\geq \frac{p_{j}R_{j}-(1-p_{j})F_{j}}{1-p_{j}}
\end{align*}
So questions should be answered in order of decreasing $$\frac{p_{i}R_{i}-(1-p_{i})F_{i}}{1-p_{i}}$$ \\

(b) With a no cost option of stopping, the questions would be answered in the same order as there would be no change to the argument above. We must now consider when the player should stop answering questions. Before each question they have two options stop answering questions or answer the next question. The player should chose the option with the higher expected reward. Clearly the expected reward for stopping is 0. The optimal stopping space would be 
$$T_{k}=\{x|0 \leq  \mathbb{E}[p_{k}R_{k}-(1-p_{k})F_{k} + J_{k+1}(x_{k}+p_{k}R_{k} ) \}$$
with 
\begin{align*}
J_{n}(x_{n}) &= 0\\
J_{k}(x_{k}) &= max[0, \mathbb{E}[p_{k}R_{k}-(1-p_{k})F_{k} + J_{k+1}(x_{k}+p_{k}R_{k} ) \}\\
\end{align*}

\end{document}















