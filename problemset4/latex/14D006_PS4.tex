\documentclass[11pt, oneside]{article}   	% use "amsart" instead of "article" for AMSLaTeX format
\usepackage{geometry}                		% See geometry.pdf to learn the layout options. There are lots.
\geometry{letterpaper}                   		% ... or a4paper or a5paper or ...
%\geometry{landscape}                		% Activate for rotated page geometry
%\usepackage[parfill]{parskip}    		% Activate to begin paragraphs with an empty line rather than an indent
\usepackage{graphicx}				% Use pdf, png, jpg, or eps§ with pdflatex; use eps in DVI mode
								% TeX will automatically convert eps --> pdf in pdflatex
								\usepackage[bb=boondox]{mathalfa}
\usepackage{mathtools}
\usepackage{amssymb}

\usepackage{amsmath,amsfonts,amsthm} % Math packages
\usepackage{bm}
\usepackage{graphicx}
\usepackage{dsfont}
\graphicspath{ {images/} }

%SetFonts

%SetFonts

\DeclareMathOperator*{\argmin}{arg\,min}
\DeclareMathOperator*{\argmax}{arg\,max}
\newcommand{\horrule}[1]{\rule{\linewidth}{#1}} % Create horizontal rule command with 1 argument of height

\title{
\normalfont \normalsize
\textsc{14D006 Stochastic Models and Optimization} \\ [25pt] % Your university, school and/or department name(s)
\horrule{0.5pt} \\[0.4cm] % Thin top horizontal rule
\huge Problemset 4\\ % The assignment title
\horrule{2pt} \\[0.5cm] % Thick bottom horizontal rule
}

\author{Daniel Bestard, Michael Cameron, Hans-Peter H{\"o}llwirth, Akhil Lohia} % Your name

\date{\normalsize\today} % Today's date or a custom date

\begin{document}
\maketitle


%%%%%%%%
% Problem 1 %
%%%%%%%%
\section{Linear-Quadratic Problem with Forecasts}
First of all let's set up the problem in order to make the proof. The dynamics of the problem is linear function of the form:
$$x_{k+1} = A_{k}x_{k} + B_{k}u_{k} + w_{k}$$
and the cost function is a quadratic function of the form:
$$\underset{w_{k}}{\mathbb{E}}\bigg\{g_{N}(x_{N}) + \sum_{k=0}^{N-1}g_{k}(x_{k},u_{k},w_{k})\bigg\}$$
where
$$g_{N}(x_{N}) = x'_{N}Q_{N}x_{N}$$
$$g_{k}(x_{k},u_{k},w_{k}) = x'_{k}Q_{k}x_{k} + u'_{k}R_{k}u_{k}$$
The matrices $A_{k}$, $B_{k}$, $Q_{k}$ and $R_{k}$ are given and the last two are positive semidefinite symmetric and positive definite symmetric, respectively.\\

The DP-algorithm that solves the minimization problem is:
$$J_{N}(x_{N}) = x'_{N}Q_{N}x_{N}$$
$$J_{k}(x_{k}) = \underset{u_{k}}{min} \underset{w_{k}|y_{k}}{\mathbb{E}} \bigg\{x'_{k}Q_{k}x_{k} + u'_{k}R_{k}u_{k} + J_{k+1}(A_{k}x_{k} + B_{k}u_{k} + w_{k})\bigg\}$$
By induction we get that:
\begin{align*}
J_{N-1}(x_{N-1}) &= \underset{u_{N-1}}{min} \underset{w_{N-1}|y_{N-1}}{\mathbb{E}} \bigg\{x'_{N-1}Q_{N-1}x_{N-1} + u'_{N-1}R_{N-1}u_{N-1} + \\
&+ \big(A_{N-1}x_{N-1} + B_{N-1}u_{N-1} + w_{N-1}\big)'Q_{N}\big(A_{N-1}x_{N-1} + B_{N-1}u_{N-1} + w_{N-1}\big) \bigg\}\\
&= x'_{N-1}Q_{N-1}x_{N-1} + x'_{N-1}A'_{N-1}Q_{N}A_{N-1}x_{N-1} +\\
&+ \underset{u_{N-1}}{min}\bigg\{ u'_{N-1}R_{N-1}u_{N-1} + u'_{N-1}B'_{N-1}Q_{N}B_{N-1}u_{N-1} + 2x'_{N-1}A'_{N-1}Q_{N}B_{N-1}u_{N-1}\bigg\} + \\
&+ \underset{w_{N-1}|y_{N-1}}{\mathbb{E}}\bigg\{ w'_{N-1}Q_{N}w_{N-1} + 2x'_{N-1}A'_{N-1}Q_{N}w_{N-1}\bigg\} + \\
&+ \underset{u_{N-1}}{min} \underset{w_{N-1}|y_{N-1}}{\mathbb{E}} \bigg\{2u'_{N-1}B'_{N-1}Q_{N}w_{N-1}\bigg\}
\end{align*}

By differentiating the previous expression and setting it to 0, we obtain the following result:
$$\big(R_{N-1} + B'_{N-1}Q_{N}B_{N-1}\big)u_{N-1}^{*} = -B'_{N-1}Q_{N}A_{N-1}x_{N-1} - B'_{N-1}Q_{N}\mathbb{E}[w_{N-1}|y_{N-1}]$$

Given the definitions provided previously we can note that the matrix $R_{N-1} + B'_{N-1}Q_{N}B_{N-1}$ is positive definite, which means that we can invert it and obtain the following optimal value:
\begin{align*}
u_{N-1}^{*} &= - \big(R_{N-1} + B'_{N-1}Q_{N}B_{N-1}\big)^{-1}\big(B'_{N-1}Q_{N}A_{N-1}x_{N-1} + B'_{N-1}Q_{N} \mathbb{E}[w_{N-1}|y_{N-1}]\big)\\
&= - \big(R_{N-1} + B'_{N-1}Q_{N}B_{N-1}\big)^{-1}B'_{N-1}Q_{N}\big(A_{N-1}x_{N-1} +\mathbb{E}[w_{N-1}|y_{N-1}]\big)
\end{align*}
Note that the previous expression is already of the form of the expression to be proved. If we substitute back the optimal value $u_{N-1}^{*}$ in $J_{N-1}(x_{N-1})$ and continue doing this we get the general expression provided in the exercise, which is:
$$\mu_{k}^{*}(x_{k},y_{k}) = - \big(R_{k} + B'_{N-1}K_{k+1}B_{k}\big)^{-1}B'_{k}K_{k+1}\big(A_{k}x_{k} + \mathbb{E}[w_{k}|y_{k}]\big) + \alpha_{k}$$
where $K_{k+1}$ is a function of $A_{k}$, $B_{k}$, $Q_{k}$ and $R_{k}$.

%%%%%%%%
% Problem 2 %
%%%%%%%%
\section{Computational Assignment on Linear-Quadratic Control}


%%%%%%%%
% Problem 3 %
%%%%%%%%
\section{Asset Selling with Offer Estimation}


%%%%%%%%
% Problem 4 %
%%%%%%%%
\section{Inventory Control with Demand Estimation}


%%%%%%%%
% Problem 5 %
%%%%%%%%
\section{Robust Dynamic Programming}
Consider a variation of the basic problem in which we do not have a probabilistic description of uncertainty. Instead, we want to find the closed-loop policy $\pi = \{\mu_0(.),..,\mu_{N-1}(.)\}$ with $\mu_k(x_k) \in U_k(x_k)$ that minimizes the maximum possible cost:
$$
J_{\pi}(x_0) = \max_{w_k \in W_k(x_k,\mu_k(x_k))} \left[ g_N(x_N) + \sum_{k=0}^{N-1} g_k(x_k,\mu_k(x_k), w_k)\right]
$$

\subsection{DP formulation}
Using the principle of optimality, the DP like recursion for this variation of the basic problem looks like:
\begin{align*}
J_{N}(x_{N}) &= g_N(x_N)\\
J_{k}(x_{k}) &= \min_{u_k \in U_k(x_k)} \max_{w_k \in W_k(x_k,\mu_k(x_k))} \left[ g_k(x_k,\mu_k(x_k),w_k) + J_{k+1}(f_k(x_k,\mu_k(x_k),w_k))\right]\\
\end{align*}
Note that, when we compare the DP formulation to the original basic problem, instead of minimizing the expected cost over $w_k$, here we minimize the maximum possible cost that can result from an action $u_k=\mu_k(x_k)$.

\subsection{Reachability of a target tube}
Now assume that at each stage $k$, the state $x_k$ must belong to a given set $X_k$. A cost structure that fits the reachability problem within the general formulation looks as follows:
\begin{align*}
J_{N}(x_{N}) &= 
\begin{cases}
0 & \text{ if } x_N \in X_N\\
1 & \text{ if } x_N \not\in X_N
\end{cases}\\
J_{k}(x_{k}) &= \min_{u_k \in U_k(x_k)} \max_{w_k \in W_k(x_k,\mu_k(x_k))} \left[ J_{k+1}(f_k(x_k,\mu_k(x_k),w_k))\right]\\
\end{align*}

The set $\bar{X}_k$, i.e. the set that we must reach at stage $k$ in order to be able to maintain the state of the system in the desired tube henceforth, can be computed recursively as follows:
\begin{align*}
\bar{X}_N &= X_N\\
\bar{X}_k &= \{x_k: \exists \mu_k(x_k) \in U_k(x_k).\forall w_k\in W_k(x_k,\mu_k(x_k)).f_k(x_k,\mu_k(x_k),w_k) \in \bar{X}_{k+1}\}\\
\end{align*}
For $x_k \in \bar{X}_k$, there must exist at least one action $u_k=\mu_k(x_k)$ in set $U_k(x_k)$ such that every possible outcome $w_k$ in the outcome set $W_k(x_k,\mu_k(x_k))$ takes us to a state $x_{k+1} = f_k(x_k,\mu_k(x_k),w_k) \in \bar{X}_{k+1}$.

\end{document}















